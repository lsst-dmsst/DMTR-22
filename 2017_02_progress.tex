\documentclass[DM,lsstdraft,toc,usenatbib]{lsstdoc}

% Package imports go here
\usepackage{amsmath}	% Advanced maths commands
\usepackage{amssymb}
\usepackage{gensymb}  % degree symbol 
\usepackage{natbib}  % bibliography 
% Local commands go here

%% Journal abbreviations
\bibliographystyle{aasjournal}

\title[PDAC progress]{PDAC Tests: February 2017}

\author{
K.~Suberlak,
\v{Z}.~Ivezi\'c,
and the PDAC team.}

\setDocRef{PDAC-0217}
\date{\today}
\setDocRevision{TBD}
\setDocStatus{draft}

\setDocAbstract{%
This document is a report of progress made in testing the Preliminary Data Access Center user interface, infrastructure, and database ingestion.  We list conducted tests commenting on the outcomes. This will help direct the focus of the PDAC development to make it user-friendly and efficient. 
}

% Change history defined here. Will be inserted into
% correct place with \maketitle
% OLDEST FIRST: VERSION, DATE, DESCRIPTION, OWNER NAME
\setDocChangeRecord{%
\addtohist{1}{2017-03-11}{Initial release. Started description of initial PDAC use.}{Krzysztof Suberlak}
\addtohist{2}{2017-03-28}{Major revision. Reordered structure.}{Krzysztof Suberlak}
}

\begin{document}

% Create the title page
% Table of contents will be added automatically if "toc" class option
% is used.
\maketitle

\section{Data overview}

PDAC contains NCSA-hosted Summer 2013 DM-stack reprocessed SDSS Stripe 82 data (called 'LSST Data' in the online portal). PDAC supports online User Interface queries, as well as direct  qserv-dax SQL queries. Concrete examples of tests outlined below are available in the full PDAC report.  


\section{Performed tests}

\begin{itemize}
	\item Tested navigation within the main User Interface (UI). The front page allows access to Catalogs (Deep Source, Deep Forced Source), or Images (Deep Coadd, Science CCD Exposure). Any query can be restricted to a certain region on the sky, that can be defined as a geometrical shape. There are four shapes available under  'Method Search' menu:  cone - 'cone' method,  square - 'box' method,  ellipse - 'elliptical' method, polygon - 'polygon' method.  There are two other search methods: one that would perform a cone query against multiple objects - 'Multi-Object' method, and  another  that would search an entire available dataset without spatial constraints  - 'All Sky'. The latter two are not yet implemented. The four geometrical methods work as expected, restricting the search region to a cone, square, ellipse or polygon, centered on the chosen position.  The position on which search region is centered can be entered as (ra,dec), or NED-resolvable name. Currently examples of queries underneath the 'position' box represent regions not present in the available dataset (eg. 'm81' 'ngc 13' '12.34 34.89' '46.53, -0.251 gal'  '19h17m32s 11d58m02s equ j2000' '12.3, 8.5 b1950'). The (ra,dec) allowed by the UI are only within  $(0-360, -90,90 \degree)$ range, but the SQL query against QSERV  accepts (ra,dec) within $(-360,360 \degree)$ range, with a built-in ability to interpret negative right ascenscion.

	\item Tested the query result interface. It displays the coverage : a postage stamp miniature image of the searched region, a table view of the queried catalog, and a two-dimensional plot of two chosen columns. The postage  stamp image is not always centered on the queried image (as compared to eg. Aladin or SDSS DR13 SkyServer ), nor does it display the underlying S82 S13 images. Instead, the underlying algorithm uses images  from IRAS, DSS, 2MASS, or WISE, depending on the size of the region. In particular, beyond search radius of  approximately 1000 arcsec the postage stamp miniature shifts from SDSS-like DSS to more blurry WISE, 2MASS or IRAS.  Currently there is no information displayed about which sky survey the postage stamp image originates. The problem with image centering persists when navigating back in the web browser and entering new search coordinates or changing the search radius. The postage stamp coverage image is centered correctly if the main page of the UI is refreshed. 

	
    \item Tested the database ingestion. The S82 Summer 2013 LSST-Stack reprocessed dataset is stored at NCSA, with a local copy at the University of Washington.  The PDAC-stored dataset should be identical. We downloaded light curves for individual objects. First approach was to fold the g-band light curves of selected RR Lyrae stars on the known period.  We assumed as ground truth the periods from \cite{sesar2010}.  We found that of 343  RR Lyrae stars from \cite{sesar2010} that were within PDAC database,  all were well represented by the 'true' period.  The process of obtaining calibrated light curves was not straightforward  - following a detailed guide on PDAC Sample Queries confluence page,  to calculate magnitudes from fluxes (uncorrected for extinction), it was necessary to  make a join between \verb|RunDeepForcedSource| and \verb|Science_CCD_Exposure| catalogs.  Queries were conducted by  a python script that transfers curl subprocesses to the terminal of a user laptop, communicating with the qserv  at \url{http://lsst-qserv-dax01.ncsa.illinois.edu:5000/db/v0/tap/sync}

 \end{itemize}   

%%%%%%%%%%%%%%%%%%%% REFERENCES %%%%%%%%%%%%%%%%%%

%\bibliographystyle{apj}
\bibliography{references} 

\end{document}