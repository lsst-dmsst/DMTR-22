% mnras_template.tex
%
% LaTeX template for creating an MNRAS paper
%
% v3.0 released 14 May 2015
% (version numbers match those of mnras.cls)
%
% Copyright (C) Royal Astronomical Society 2015
% Authors:
% Keith T. Smith (Royal Astronomical Society)

% Change log
%
% v3.0 May 2015
%    Renamed to match the new package name
%    Version number matches mnras.cls
%    A few minor tweaks to wording
% v1.0 September 2013
%    Beta testing only - never publicly released
%    First version: a simple (ish) template for creating an MNRAS paper

%%%%%%%%%%%%%%%%%%%%%%%%%%%%%%%%%%%%%%%%%%%%%%%%%%
% Basic setup. Most papers should leave these options alone.
\documentclass[fleqn,usenatbib]{mnras} % a4paper,

% MNRAS is set in Times font. If you don't have this installed (most LaTeX
% installations will be fine) or prefer the old Computer Modern fonts, comment
% out the following line
\usepackage{newtxtext,newtxmath}
% Depending on your LaTeX fonts installation, you might get better results with one of these:
%\usepackage{mathptmx}
%\usepackage{txfonts}

% Use vector fonts, so it zooms properly in on-screen viewing software
% Don't change these lines unless you know what you are doing
\usepackage[T1]{fontenc}
\usepackage{ae,aecompl}

\usepackage{natbib}

%%%%% AUTHORS - PLACE YOUR OWN PACKAGES HERE %%%%%

% Only include extra packages if you really need them. Common packages are:
\usepackage{graphicx}	% Including figure files
\usepackage{amsmath}	% Advanced maths commands
\usepackage{amssymb}	% Extra maths symbols

%%%%%%%%%%%%%%%%%%%%%%%%%%%%%%%%%%%%%%%%%%%%%%%%%%

%%%%% AUTHORS - PLACE YOUR OWN COMMANDS HERE %%%%%

% Please keep new commands to a minimum, and use \newcommand not \def to avoid
% overwriting existing commands. Example:
%\newcommand{\pcm}{\,cm$^{-2}$}	% per cm-squared

%%%%%%%%%%%%%%%%%%%%%%%%%%%%%%%%%%%%%%%%%%%%%%%%%%

%%%%%%%%%%%%%%%%%%% TITLE PAGE %%%%%%%%%%%%%%%%%%%

% Title of the paper, and the short title which is used in the headers.
% Keep the title short and informative.
\title[PDAC report]{Preliminary Data Access Center : User Report}

% The list of authors, and the short list which is used in the headers.
% If you need two or more lines of authors, add an extra line using \newauthor
\author[K. Suberlak et al. ]{
Krzysztof Suberlak $^{1}$\thanks{E-mail: suberlak@uw.edu (KS)}
\v{Z}eljko Ivezi\'c, $^{1}$
\\
% List of institutions
$^{1}$Department of Astronomy, University of Washington, Seattle, WA, United States\\
}

% These dates will be filled out by the publisher
\date{Accepted XXX. Received YYY; in original form ZZZ}

% Enter the current year, for the copyright statements etc.
\pubyear{2017}

% Don't change these lines
\begin{document}
\label{firstpage}
\pagerange{\pageref{firstpage}--\pageref{lastpage}}
\maketitle

% Abstract of the paper
\begin{abstract}
A report on user experience of  the Preliminary Data Access Center (PDAC).  Employing the SDSS and GAIA datasets we test the quality and ease of access to the data. PDAC will pave the way to the Science User Interface and Tools  (SUIT).   We employ both in-detail study of individual objects, and a statistical study of an ensemble of objects. We evaluate user-friendliness of the current interface, and make recommendations for  its future improvements. 
\end{abstract}

% Select between one and six entries from the list of approved keywords.
% Don't make up new ones.
% XXX  :  these are made up.  Remove this for now ...
%\begin{keywords}
%SUIT,  PDAC,  data access
%\end{keywords}

%%%%%%%%%%%%%%%%%%%%%%%%%%%%%%%%%%%%%%%%%%%%%%%%%%

%%%%%%%%%%%%%%%%% BODY OF PAPER %%%%%%%%%%%%%%%%%%

\section{Introduction}

This is a document to report on the user experience testing of the Preliminary Data Access Center.  The Large Scale Synoptic Telescope (LSST)  will  produce a big volume of data. Such unprecedented data stream poses new challenges  to provide an easy access for users, in such a way that they can quickly find what they need, and thus be able to focus on the science goal that they would like to achieve.  The detail description of  such online user-interface called Science User Interface and Tools is outlined in documents LDM-130 (SUIT requirements)  and LDM-492  (SUIT Vision).  
An idea of having an interface to the data is not new : there exists Aladin,  SDSS CAS jobs,  IPAC  IRSA,  Mikulsky NASA Archive, NED, and many other archives. These allow a user to query for data (either via SQL query, or interface), returning the data table. Some user interfaces (eg. IRSA) have some rudimentary plotting capabilities.  There have been ideas of a new  interface, that would not only eg. plot the lightcurve and display the spectrum,  but also allow the user to run some machine learning algorithms, or simple models that can help narrow down the query, or obtain science results in the browser.  Namely, Victor Pankratius, from MIT, in his talk   "Computer-Aided Discovery: Towards Scientific Insight Generation with Machine Support"  outlined the idea of an ipython notebook - access to data, which lives in the cloud, is allocated some CPU  share and memory,  and allows one to upload / download the data and run the model in real time, which is especially helpful to geoscientists doing fieldwork, where new data acquisition conditions their next step.  

Indeed, astronomers may find that quick look into the data, finding eg. all stars that exhibit RR Lyr variability and have been observed in a certain region of the sky, is very helpful.  

Here we outline the user experience of PDAC (see PDAC technical description on \footnote{\url{https://confluence.lsstcorp.org/display/DM/Guide+to+PDAC+version+1}} 

Currently, PDAC v1, under tab 'LSST Data' in the upper-left corner of the interface (see Fig.~\ref{fig:PDAC_interface}) includes the Summer 2013 DM-stack reprocessed SDSS Stripe 82 data, hosted at the NCSA on the LSST prototype ("integration cluster") hardware, in Qserv [Gregory Dubois-Felsmann, priv.comm. 02-20-2017, slack].  The reprocessing included: 
\begin{itemize}
\item coadding the data from all epochs in each of the ugriz SDSS filters. Measurements on coadds (per object) are available as  \verb|RunDeepSource| table, accessible via Catalogs -->   'DeepSource' .  The single-band coadded images with MariaDB metadata are available as \verb|DeepCoadd| table, accessibla via Images -->  'DeepCoadd' . 
\item using i-band detections to seed forced photometry on all epochs in all bands. The results of photometry are available as \verb|RunDeepForcedSource| table, accessible via  Catalogs --> 'Deep Forced Source' .  
\item For reference , the individual calibrated single epoch images are available as \verb|Science_Ccd_Exposure| table, accessible via Images --> 'Science CCD Exposure'   
\end{itemize}


\begin{figure*}
\includegraphics[width=\textwidth]{1_PDAC_interface}
\caption{The main user interface of PDAC ver. 1 }
\label{fig:PDAC_interface}
\end{figure*}





Details of  the S82 LSST reprocessing can be found in the PDAC document \url{https://confluence.lsstcorp.org/display/DM/Properties+of+the+2013+SDSS+Stripe+82+reprocessing}. Additional details of the schema are also outlined in the LSST Data Challenge Report [Shaw, Juric, Becker, Krughoff et al. 2013], and the LSST Database Schema Browser~\footnote{\url{https://lsst-web.ncsa.illinois.edu/schema/index.php?t=DeepForcedSource&sVer=S12_lsstsim}}. 

PDAC v1 under tab 'External Catalogs' also provides access to  all NASA/IPAC Infrared Science Archive(IRSA) publicly accessible catalogs, including GAIA, WISE, etc. (see Fig.~\ref{fig:PDAC_external_cat}). These are stored at Infrared Processing and Analysis Center (IPAC) \url{http://www.ipac.caltech.edu/project/lsst}.


\begin{figure}
\includegraphics[width=\columnwidth]{2_PDAC_externals}
\caption{IPAC- hosted catalogs , accessible via IRSA. }
\label{fig:PDAC_external_cat}
\end{figure}



\section{Methods}

We perform single-object tests and statistical tests on an ensemble of objects . 

First, we study in detail a particular source - we consider examples of variable objects, confirmed by previous studies (eg. RR Lyrae from Sesar+2010, Table 1). We download these from the S82 dataset on PDAC, run Lomb-Scargle periodogram to find period, and plot the phased lightcurve. \cite{sesar2010}  performed lightcurve template fits to 483 RR Lyrae lightcurves from SDSS (see Fig.\ref{fig:RRLyr_distribution}. Both fit parameters and lightcurves are publicly accessible in the online version of the journal%\footnote{\url{http://iopscience.iop.org/article/10.1088/0004-637X/708/1/717/meta#apj326724t2}}. 

\begin{figure}
\includegraphics[width=\columnwidth]{Fig_1_Sesar_2010_RRLyr}
\caption{ Distribution of RR Lyrae periods  for 483 objects in \citep{sesar2010}. Note the bimodal distribution, reflecting two main RR Lyrae types : 309 RRab (right) and 104 RRc (left) (see also Fig.16 in \citep{sesar2010}).}
\label{fig:RRLyr_distribution}
\end{figure}
% why using cite here throws 

Comparing the S82 data stored at PDAC to the data from \cite{sesar2010},  we want to treat the latter as 'ground truth', but as a sanity check we perform Lomb Scargle periodogram testing to confirm the more detailed analysis of \cite{sesar2010}. Using {\it astroML} python module \citep{astroML}, we sample the uniformly spaced frequency grid with N=5000 samples span between the smallest and the largest frequency reported in Table 1 of \cite{sesar2010} $\pm 10\%$, i.e. $\omega_{min} = 0.9 ( 2 \pi / P_{min})$, $\omega_{max} = 1.1 ( 2 \pi / P_{min})$. We use the default {\it astroML} Lomb Scargle periodogram settings, namely generalized LS (see Eq.20 in \cite{zechmeister2009}, and Section 10.3.2 in \cite{ivezic2014}).   



\begin{figure*}
\includegraphics[width=\textwidth]{Fig_2_RRLyr_LS_periodogram_ID_13350_g}
\caption{ An example of the  {\it astroML} Lomb Scargle periodogram performance,  calculated for RR Lyr ID=13350  in SDSS g band (following Table 2 in \citep{sesar2010}). It took 18.6 miliseconds on a laptop to calculate this periodogram. The upper left panel depicts the raw  SDSS lightcurve data. The upper right panel shows the phased lightcurve constructed with a cited period of 0.547987 days ('$P_{true}$'). The lower left panel shows the Lomb Scargle periodogram, where the orange and magenta vertical lines mark the location of the highest periodogram peak, and the frequency based on the reported period ($\omega_{true} = 2 \pi / P_{true}$). The lower right panel shows the phased lightcurve constructed with the Lomb-Scargle Periodogram period of 0.547161 days,  corresponding to the highest peak, $P_{fit} = 2 \pi / \omega_{fit}$. The horizontal red and green lines mark the $5\%$ and $1\%$  significance levels for the highest peak, as found from 500 bootstrap resamplings ( See \url{http://www.astroml.org/book\_figures/chapter10/index.html}) }
\label{fig:RRLyr_periodogram_good}
\end{figure*}
%


Using the same frequency grid for all 483 RR Lyrae, we compute Lomb-Scargle periodograms, and determine the best-fit period from the highest frequency peak (see Fig.~\ref{fig:RRLyr_period_ratios}).  We find that for about half of the lightcurves the Lomb-Scargle periodogram fitting single-term Fourier Series  is sufficient to find the right period (note middle group  centered on 1 on Fig.~\ref{fig:RRLyr_period_ratios}, and an example on Fig.~\ref{fig:RRLyr_periodogram_good} ). However, there are many cases  where the naive single sinusoid is insufficient to correctly fit the period (groups outside of 1 on Fig.~\ref{fig:RRLyr_periodogram_good})


\begin{figure}
\includegraphics[width=\columnwidth]{Fig_3_RRLyr_omega_ratio}
\caption{The distribution of the ratio of $\omega_{true}$ to $\omega_{fit}$, where $\omega_{true}$ is inferred directly from the 'ground truth' - period cited in Table 2 of \citep{sesar2010}. We take the same SDSS data from the paper (Table 1 in \citep{sesar2010}), and calculate the Lomb-Scargle single-term generalized periodogram. The frequency corresponding to the highest peak is  $\omega_{fit}$. Thus, wherever  this ratio is approximately equal to 1, this means that the naive  LS approach is able to recover the 'true' period.  However,  where the highest  frequency peak is not the same as $\omega_{true}$, the ratio will be smaller or bigger from 1. This may be caused by the inherent simplicity of the simple single-term Fourier Series fitting. Indeed, some RR Lyrae lightcurves may have shapes that are insufficiently described by a single sinusoid (as on Fig.10.18  in \citep{ivezic2014}). }
\label{fig:RRLyr_period_ratios}
\end{figure}


\begin{figure*}
\includegraphics[width=\textwidth]{Fig_2_RRLyr_LS_periodogram_ID_4099_g}
\caption{ A failure of naive single Lomb Scargle periodogram performance - here the ratio of $\omega_{true}$ to $\omega_{fit}$ equals 0.437. Top-left : the raw  SDSS lightcurve data for RR Lyr ID=4099 in g band. Top-right : the phased lightcurve constructed with a cited period of 0.641754 days ('$P_{true}$'). Bottom-left: the Lomb Scargle periodogram on  a uniform frequency grid (5000 bins), with the orange and magenta vertical lines marking the location of the highest periodogram peak, and the frequency based on the reported period ($\omega_{true} = 2 \pi / P_{true}$). Note that $\omega_{fit}$ and $\omega_{true}$ significantly differ for this RR Lyr,  and the 'true' frequency, backed-up by the full lightcurve fitting of \citep{sesar2010}, appears as only one of insignificant periodogram peaks. As on Fig.~\ref{fig:RRLyr_periodogram_good},  the horizontal red and green lines mark the $5\%$ and $1\%$  significance levels for the highest peak, as found from 500 bootstrap resamplings.  Bottom-right : the phased constructed with the $P_{fit}$ of  0.280827 days. }
\label{fig:RRLyr_periodogram_fail1}
\end{figure*}
%

\begin{figure*}
\includegraphics[width=\textwidth]{Fig_2_RRLyr_LS_periodogram_ID_470994_g}
\caption{Same as Fig.~\ref{fig:RRLyr_periodogram_fail1}, with $\omega_{true} / \omega_{fit} = 1.53$. Here  RR Lyr ID=470994 has a cited period of 0.346794  days ('$P_{true}$'), whereas period derived from the Lomb-Scargle periodogram is 0.531667.}
\label{fig:RRLyr_periodogram_fail2}
\end{figure*}

Using the Ra, dec for the RR Lyrae we positionally query the PDAC  \verb|RunDeepForcedSource| database to find objects within 2 arcsec radius. For these, we obtain calibrated g-magnitude lightcurves querying the \verb|RunDeepForcedSource| and \verb|Science_Ccd_Exposure| for the zero point magnitudes per exposure. Exactly as for \cite{sesar2010} SDSS lightcurves considered before, for PDAC S82 lightcurves we also calculate Lomb-Scargle periodogram and find the most-significant frequency (fit the best period). On Fig.~\ref{fig:RRLyr_period_ratios_PDAC} we consider the ratio of  $\omega_{PDAC}$ to $\omega_{true}$, analoguosly to Fig.~\ref{fig:RRLyr_period_ratios}



\begin{figure*}
\includegraphics[width=\textwidth]{Fig_2_RRLyr_LS_periodogram_ID_13350_gPDAC}
\caption{ The same object as Fig.~\ref{fig:RRLyr_periodogram_good}, but using data downloaded using PDAC. Using PDAC data, the RR Lyr ID=13350 has a best-fit period of 0.547969 days, almost identical to the period found by \citep{sesar2010} of 0.547969 days. Top-right panel shows the phased PDAC data lightcurve folded on the 'true' period, and top-right : on the 'best-fit' period. Everything else as on Fig.~\ref{RRLyr_periodogram_good}}
\label{fig:RRLyr_periodogram_PDAC}
\end{figure*}



\begin{figure}
\includegraphics[width=\columnwidth]{Fig_4_RRLyr_omega_ratio_PDAC}
\caption{The distribution of the ratio of $\omega_{true}$ to $\omega_{PDAC}$.  $\omega_{true}$ is inferred directly from the 'ground truth' - period cited in the Table 2 of \citep{sesar2010}, that used SDSS data for S82.  We queried positionally against the same objects the PDAC version of SDSS reprocessed forced photometry on S82. We call this dataset 'PDAC'. For each RR Lyr star we obtained calibrated g-band  lightcurves, that were used to calculate the Lomb-Scargle periodogram, on an identical frequency grid to that used previously on SDSS data from Table 1 in \citep{sesar2010}. Based on the highest peak in each periodogram we found the best-fit frequency, $\omega_{PDAC}$.  The ratio of $\omega_{true}$ to $\omega_{PDAC}$,  analoguously to $\omega_{true}$ to $\omega_{fit}$, represents how well can naive Lomb-Scargle approach reproduce periods found by \citep{sesar2010} by more sophisticated lightcurve template-fitting methods. [NOTE : this is an incomplete dataset! not all RR Lyr LCs are available from PDAC --> some RA, Dec have no matching PDAC object within 2 arcsecs ! ]}
\label{fig:RRLyr_period_ratios_PDAC}
\end{figure}


We tested the periodogram results for few RR Lyrae using the NASA Exoplanet Archive Periodogram\footnote{http://exoplanetarchive.ipac.caltech.edu/cgi-bin/Pgram/nph-pgram}. The calculation for RR Lyr ID=13350  in g-band took 15 seconds (illustrated on Fig.~\ref{fig:exoplanet_archive}). That RR Lyr was well described by a single Fourier component - we also check an RR Lyr ID 4099 (true period 0.641754 days, SDSS LS 0.280827, PDAC LS  ) : period 0.64175029 , and ID  470994 (true period 0.346794 days, SDSS LS 0.531667 , PDAC LS )
[MAKE A TABLE comparing SDSS-Sesar, SDSS-LS, PDAC-LS,  SDSS-Exoplanet, PDAC-Exoplanet] 

\begin{figure*}
\includegraphics[width=\textwidth]{Exoplanet_archive_periodogram}
\caption{ The same object as Fig.~\ref{fig:RRLyr_periodogram_good}, and Fig~\ref{fig:RRLyr_periodogram_PDAC},  using  the SDSS data from \citep{sesar2010}.  The highest significance frequency peak (power 21.58) corresponds to  a period of 0.35365194 days. Only the second in significance peak (power 20.62) corresponds to the 'true' period of 0.547969 \citep{sesar2010}. Note the bottom-left corner : the calculation took 15 secs for one lightcurve (compare to few miliseconds of AstroML code naive single-sinusoid approach that gave the same result for this particular object)/  }
\label{fig:exoplanet_archive}
\end{figure*}



Second, we query the S82 database against a small subset of a given S82 patch (few degrees), downloading lightcurves for $\sim100 000 $ objects in that area of the sky. We plot color-color diagrams, as in \cite{sesar2007}, Fig.3 ,4, and color - magnitude diagrams to show the morphology of the Sgr dSph tidal stream \citep{sesar2010}.  





\section{Results}

\section{Conclusions}

\section*{Acknowledgements}
Thank you ! 

%%%%%%%%%%%%%%%%%%%%%%%%%%%%%%%%%%%%%%%%%%%%%%%%%%

%%%%%%%%%%%%%%%%%%%% REFERENCES %%%%%%%%%%%%%%%%%%

% The best way to enter references is to use BibTeX:

\bibliographystyle{apj}
\bibliography{references} % if your bibtex file is called example.bib


%%%%%%%%%%%%%%%%%%%%%%%%%%%%%%%%%%%%%%%%%%%%%%%%%%

%%%%%%%%%%%%%%%%% APPENDICES %%%%%%%%%%%%%%%%%%%%%

\appendix

\section{Some extra material}

If you want to present additional material which would interrupt the flow of the main paper,
it can be placed in an Appendix which appears after the list of references.

%%%%%%%%%%%%%%%%%%%%%%%%%%%%%%%%%%%%%%%%%%%%%%%%%%


% Don't change these lines
\bsp	% typesetting comment
\label{lastpage}
\end{document}

% End of mnras_template.tex