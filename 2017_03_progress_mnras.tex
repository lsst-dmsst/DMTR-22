% mnras_template.tex
%
% LaTeX template for creating an MNRAS paper
%
% v3.0 released 14 May 2015
% (version numbers match those of mnras.cls)
%
% Copyright (C) Royal Astronomical Society 2015
% Authors:
% Keith T. Smith (Royal Astronomical Society)

% Change log
%
% v3.0 May 2015
%    Renamed to match the new package name
%    Version number matches mnras.cls
%    A few minor tweaks to wording
% v1.0 September 2013
%    Beta testing only - never publicly released
%    First version: a simple (ish) template for creating an MNRAS paper

%%%%%%%%%%%%%%%%%%%%%%%%%%%%%%%%%%%%%%%%%%%%%%%%%%
% Basic setup. Most papers should leave these options alone.
\documentclass[fleqn,usenatbib, onecolumn]{mnras} % a4paper,

% to add SQL queries used in appendix 
\usepackage{xcolor,listings}

% MNRAS is set in Times font. If you don't have this installed (most LaTeX
% installations will be fine) or prefer the old Computer Modern fonts, comment
% out the following line
\usepackage{newtxtext,newtxmath}
% Depending on your LaTeX fonts installation, you might get better results with one of these:
%\usepackage{mathptmx}
%\usepackage{txfonts}

% Use vector fonts, so it zooms properly in on-screen viewing software
% Don't change these lines unless you know what you are doing
\usepackage[T1]{fontenc}
\usepackage{ae,aecompl}

\usepackage{natbib}

%%%%% AUTHORS - PLACE YOUR OWN PACKAGES HERE %%%%%

% Only include extra packages if you really need them. Common packages are:
\usepackage{graphicx}	% Including figure files
\usepackage{amsmath}	% Advanced maths commands
\usepackage{amssymb}	% Extra maths symbols
\usepackage{gensymb}
%%%%%%%%%%%%%%%%%%%%%%%%%%%%%%%%%%%%%%%%%%%%%%%%%%

%%%%% AUTHORS - PLACE YOUR OWN COMMANDS HERE %%%%%

% Please keep new commands to a minimum, and use \newcommand not \def to avoid
% overwriting existing commands. Example:
%\newcommand{\pcm}{\,cm$^{-2}$}	% per cm-squared

%%%%%%%%%%%%%%%%%%%%%%%%%%%%%%%%%%%%%%%%%%%%%%%%%%

%%%%%%%%%%%%%%%%%%% TITLE PAGE %%%%%%%%%%%%%%%%%%%

% Title of the paper, and the short title which is used in the headers.
% Keep the title short and informative.
\title[PDAC progress]{PDAC Tests: March 2017}

% The list of authors, and the short list which is used in the headers.
% If you need two or more lines of authors, add an extra line using \newauthor
\author[K. Suberlak et al. ]{
Krzysztof Suberlak, $^{1}$\thanks{E-mail: suberlak@uw.edu (KS)}
\v{Z}eljko Ivezi\'c, $^{1}$ and the PDAC team  
\\
% List of institutions
$^{1}$Department of Astronomy, University of Washington, Seattle, WA, United States\\
}

% These dates will be filled out by the publisher
\date{Accepted XXX. Received YYY; in original form ZZZ}

% Enter the current year, for the copyright statements etc.
\pubyear{2017}

% Don't change these lines
\begin{document}
\label{firstpage}
\pagerange{\pageref{firstpage}--\pageref{lastpage}}
\maketitle

% Abstract of the paper
\begin{abstract}
A report of progress made in testing the Preliminary Data Access Center user interface, infrastructure, and database ingestion.  We list conducted tests commenting on the outcomes. This will help direct the focus of the PDAC development to make it user-friendly and efficient. 
\end{abstract}


%%%%%%%%%%%%%%%%%%%%%%%%%%%%%%%%%%%%%%%%%%%%%%%%%%

%%%%%%%%%%%%%%%%% BODY OF PAPER %%%%%%%%%%%%%%%%%%

\section*{Introduction}

\subsection*{Data overview}
We continued testing the structure of the S82 database ingested in PDAC,  performed tests of individual lightcurves, and Box queries to test source density. We started investigating the periodogram tool and ways in which the database can be made more user-friendly through improved  metadata.  We also tried ways of going beyond the built - in functions to analyse the data with user-defined functions. 


\subsection*{Performed tests}

\begin{itemize}
	\item Tested whether it is possible to run code parallel to the database server,  or locally to the data (at the NCSA), using a jupyter notebook.  This would be helpful in the analysis of forced photometry, such as using the Bayesian faint flux pipeline developed at the University of Washington. In a discussion we found that the database level (qserv, scisql) User-Defined Functions (UDF) are not yet implemented. UDFs, and ability to run code next to shard server is  FY  2019-2020 specification.  The ability to run jupyter notebooks locally to the data (within the same computer cluster at eg. NCSA where the data is stored for current implementation of PDAC) is not yet implemented, but is projected to be available towards the end of FY 2017.

	\item  Tested source density using Box queries. We found that to properly compare the PDAC data to the Deep Source catalog, locally stored at the UW, we need to account for the magnitude cutoff made to create the UW catalog ( i < 23.5 mag ). We found that a straightforward SQL query was illegal due to limitations of the server in understanding the alias. The solution was to use HAVING rather than WHERE clause. This issue also uncovered a degradation of quality of error messages as they are transferred from server to server ( mariaDB to qserv, etc. )

    \item  Tested the raw lightcurve ingestion to the database. We queried against a given position, and compared the raw data stored at PDAC to the UW-stored data.  We found that direct comparison is not possible due to lack of documentation about the units of \verb|flux_psf| column in RunDeepForcedSource Catalog of S82 data. Converting the PDAC data with the scisql  functions to calibrated magnitudes we find that all the negative forced photometry measurements  (where \verb|flux_psf|$<0$) are by default discarded by the scisql function. 


 \end{itemize}   


%%%%%%%%%%%%%%%%%%%%%%%%%%%%%%%%%%%%%%%%%%%%%%%%%%

%%%%%%%%%%%%%%%%%%%% REFERENCES %%%%%%%%%%%%%%%%%%

% The best way to enter references is to use BibTeX:

\bibliographystyle{apj}
\bibliography{references} % if your bibtex file is called example.bib


% Don't change these lines
\bsp	% typesetting comment
\label{lastpage}
\end{document}

% End of mnras_template.tex